\documentclass{article}
\usepackage[utf8]{inputenc}
\usepackage{amsmath}
\usepackage{natbib}
\usepackage{hyperref}

\title{Complex Compilation Chain Example}
\author{tpmgr User}
\date{\today}

\begin{document}
\maketitle

\section{Introduction}
This project demonstrates tpmgr's compilation chain functionality, supporting multi-step compilation processes including BibTeX.

\section{Literature Citations}
According to research \citep{example2023}, LaTeX package management tools can significantly improve document writing efficiency.

Another study \cite{smith2022} demonstrates the importance of automated tool chains for academic writing.

\section{Conclusion}
This document successfully demonstrates:
\begin{itemize}
    \item Multi-step compilation chain configuration
    \item Usage of magic variables
    \item Automatic package dependency detection
\end{itemize}

\bibliographystyle{plain}
\bibliography{src/references}

\end{document}
